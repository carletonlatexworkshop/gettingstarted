% Template for Carleton College LaTeX examples
% This work is licensed under the Creative Commons Attribution 4.0 International License.
% To view a copy of this license, visit http://creativecommons.org/licenses/by/4.0/ or send a letter to Creative Commons, 444 Castro Street, Suite 900, Mountain View, California, 94041, USA.
\documentclass{article}
\usepackage{cctexexample}

\title{Getting started with \LaTeX{}}
\author{Carleton College \LaTeX{} workshop}
\date{}

\newcommand*{\code}[1]{\texttt{#1}}
\newcommand*{\filename}[1]{\texttt{#1}}
\newcommand*{\inst}[1]{\textbf{#1}}

\DeclareMathOperator{\dd}{d}

\usepackage{marvosym}

\begin{document}
\maketitle

Hello, and welcome to \LaTeX{}!
If you're just hearing about \LaTeX{} for the first time, you should read this document once through from beginning to end before you do anything with your computer.
Otherwise, you may want to skip into specific sections.
\begin{itemize}
\item
  \Cref{s:prelim} is a brief introduction to the philosophy and core concepts of \LaTeX{} as a system for producing documents.
\item
  \Cref{s:yourcomp} contains instructions for getting up and running with the \LaTeX{} software so you can write documents.
\item
  \Cref{s:firstdoc} describes how to write a minimal compileable document and make a PDF out of it.
\item
  \Cref{s:firstpset} expands on \cref{s:firstdoc} to walk you through writing up your first problem set in \LaTeX{}.
\end{itemize}

\section{Preliminaries}
\label{s:prelim}
You've probably used a word processor (like Microsoft's Word or OpenOffice Writer) to produce documents in the past.
\LaTeX{} is also a way to produce documents, but it's very different.

A typical \LaTeX{} document looks something like this:
\begin{lstlisting}[frame=single,caption=Basic \LaTeX{} example,label=lst:basicex, language=TeX]
  \documentclass{article}
  \title{Article title}
  \author{Author name}
  \begin{document}
  \maketitle
  This is an example of a document file.
  \end{document}
\end{lstlisting}

It's code!
\LaTeX{} is a \emph{structured markup language}, sort of like HTML; you build a document by writing some text which is \enquote{marked up} with information which describes that text's structure.
This code is the interpreted by a computer program which renders it into a nicely typeset PDF.

\LaTeX{} has a lot of great advantages:
\begin{itemize}
\item
  It's great at handling complex typesetting tasks like mathematical formulas.
  Check this out:
  \begin{equation}
    R = \frac{ \sum_{i=1}^{n} \pbrac{x_{i} - \bar{x}} \pbrac{y_{i} - \bar{y}} }{ \sqrt{\sum_{i = 1}^{n} \pbrac{x_{i} - \bar{x}}^{2} \sum_{i = 1}^{n} \pbrac{y_{i} - \bar{y}}^{2}} }
    \label{eq:statsthing}
  \end{equation}
  or this:
  \begin{equation}
    \label{eq:greens}
    \iint_{\Sigma} \cbrac*{ \pbrac*{ \frac{\partial R}{\partial y} - \frac{\partial Q}{\partial z}} \dd y \dd z +\pbrac*{\frac{\partial P}{\partial z} - \frac{\partial R}{\partial x}} \dd z \dd x  + \pbrac*{\frac{\partial Q}{\partial x} - \frac{\partial P}{\partial y}} \dd x \dd y}
    = \oint_{\partial \Sigma} \cbrac{P \dd x + Q\dd y + R \dd z}
  \end{equation}
  Pretty slick, huh?

\item
  It separates content from style.
  For example, to start a new section of a document, you just type \verb|\section{Name of section}|, and to produce text which is \emph{emphasized}, you just type \verb|\emph{emphasized}|.
  This means that you never have to worry about remembering how to style any particular element, and it makes writing documents \emph{much faster} once you get the hang of it.

\item
  It produces beautiful documents.
  The underlying \TeX{} typesetting engine is extremely powerful and professional-grade, producing output light-years ahead of what's typical from word processors.
  Your eyes will thank you!

\item
  It has powerful cross-referencing and other document tools.
  For example, I can refer to \cref{eq:greens} above simply by typing \verb|\label{eq:greens}| in the code for the equation and then \verb|\ref{eq:greens}| down here.
  The numbering is handled automatically, so I can move things around or even change the numbering scheme without breaking it.

\item
  It's highly portable.
  The source code you write is plain text, so you can email it, post it on a website, or even print it out and type it back in without messing anything up.
  The rendered documents are PDF, which will look the same on any computer in the world.
  No more worrying about what versions of Word everyone in your group is using!
\end{itemize}

Of course, it also has some disadvantages and costs.
\begin{itemize}
\item
  \enquote{Compiling} \LaTeX{} source into the finished PDF requires specialized software.
  We'll end up delegating this task to \enquote{the cloud}, as we'll see shortly.

\item
  It's hard to use \LaTeX{} for page layout.
  It will format your documents the way it thinks is best, and you're going to have a bad time if you try to tell it what to do.
  Go with the flow and focus on your content; your documents will look great!
  (If you have specific page-layout needs, like a newsletter or event poster, \LaTeX{} probably isn't the right software for that task.)

\item
  You don't know how to use it.
  \textbf{Yet.}
\end{itemize}

\section{Getting your computer set up for \LaTeX{}}
\label{s:yourcomp}
The version of a \LaTeX{} document that you work on will consist of source code, similar to the HTML code for a web page or the source code for a computer program.
This needs to be processed (or \enquote{compiled}) to produce the finished PDF document.

Traditionally, this is done with software installed on your computer.
The lab computers in the CMC are set up with such software on both OSX and Windows, so you are welcome to use that for your \LaTeX{} work.
If you wish to do this on your own computer, you should download and install \href{http://www.tug.org/texlive/acquire-netinstall.html}{\TeX{} Live}, which includes both a \LaTeX{} compiler and the \TeX{}Works editing software.

Unfortunately, this software tends to be a bit complicated, and there's enough diversity among programs that it's very hard to write good general documentation.
Thus, for this document and the rest of the workshop, we'll use \href{http://www.writelatex.com}{Write\LaTeX{}}, a \LaTeX{} editing and compilation suite that runs on a website.
(It's the same idea as Google Docs, but for \LaTeX{} documents.)

To get started, just visit \url{http://www.writelatex.com} and click on \enquote{Create a new paper}.
You don't even have to sign up for an account, but doing so will give you access to file management and other useful features, so it's recommended.

\subsection{Write\LaTeX{} features and caveats}
\label{s:wlcaveats}
Write\LaTeX{} is a cloud service, which comes with a few important caveats.
\begin{itemize}
\item
  Your files are stored on Write\LaTeX{}'s servers, not your computer.
  If a tornado, alien attack, or fat-fingered systems administrator destroys them, you will have no recourse.
  Keep backups of anything important!

\item
  Your files are stored on Write\LaTeX{}'s servers, not your computer.
  Anyone at Write\LaTeX{} can read them (the same way Google employees can read your GMail).
  Think carefully about this before you upload compromising photos, proprietary data, or your plans for world domination.

\item
  Write\LaTeX{} will only work if you have a connection to the Internet.

\item
  If you use the free plan, your documents will be visible to and editable anyone who knows the right URL.
  This URL won't be publicly advertised (and consists of a long, random string), so you should be safe, but it's worth keeping in mind.
\end{itemize}

It also has some significant advantages over desktop-based software:
\begin{itemize}
\item
  You don't have to do anything to set it up.

\item
  You can easily share editing privileges on a document with others---to work on a problem set with classmates, for example, or to collaborate on your comps paper.

\item
  Creating a new document from the Carleton templates requires just two clicks!
\end{itemize}

\section{Your first \LaTeX{} document}
\label{s:firstdoc}
To get started, \inst{open a copy of the \href{https://www.writelatex.com/templates/carleton-problem-set-template/vjfwnwppbrzr}{Carleton problem set template}}.
This consists of two files: \code{ccpset.sty}, a \enquote{style file} that works behind the scenes to format the document, and \code{ccpset.tex}, a \enquote{\TeX{} source file} that will be compiled into a document.
Write\LaTeX{} will automatically open the document file for you.

You should see something like \cref{lst:ccpsetempty}.

\lstinputlisting[float=phtb, frame=single, caption=The Carleton problem set template, label=lst:ccpsetempty, language=TeX]{homework/ccpset.tex}

Let's take a quick tour of the important features of this file and edit it as we go.
The first part is the \emph{preamble}, which configures \LaTeX{} and tells it about your document.

\begin{itemize}
\item
  Any line that starts with a \code{\%} is a \emph{comment}.
  The \LaTeX{} compiler will ignore these lines; they're only there to be read by humans.

\item
  Anything that starts with a \code{\textbackslash} is a \emph{command}\footnote{Note that this implies that you can't get a \textbackslash{} just by typing \enquote{\code{\textbackslash{}}}! \code{\textbackslash{}} is a \emph{control character}, which means that \LaTeX{} treats it specially. We'll see more of these later.}.
  Each command can take some \emph{arguments}, which are contained in curly braces (\code{\{} and \code{\}}), and possibly an \emph{optional argument}, which is contained in square braces (\code{[} and \code{]}).

\item
  The command \code{\textbackslash{}documentclass} on the first line sets the document's \emph{class}; in this case, it is set to \code{article}.
  Document classes are used to set up things like page size, pagination, header formatting, and other large-scale things.
  The optional argument \code{twoside} tells \code{article} to set things up for double-sided printing, which changes how margins are set up.

\item
  The command \code{\textbackslash{}usepackage} tells \LaTeX{} to load a \emph{package}, which can execute lots of code on your behalf.
  The \code{ccpset} package configures things for a Carleton problem set, and the \code{lmodern} package fixes some problems with the default \LaTeX{} fonts.

\item
  The commands \code{\textbackslash{}title}, \code{\textbackslash{}author}, \code{\textbackslash{}date}, and \code{\textbackslash{}prof} set the information that will be used to make the document's title header.
  \inst{Go ahead and replace their arguments with some of your own; for example, replace \enquote{Author} with your own name.}
  Be sure to leave the opening (\code{\{}) and closing (\code{\}}) braces!
\end{itemize}

The second part of the file is the \emph{body}, where you write the actual text of your document.
\begin{itemize}
\item
  The entire body is wrapped in the \enquote{document} \emph{environment}, which starts with \code{\textbackslash{}begin\{document\}} and ends with \code{\textbackslash{}end\{document\}}.

\item
  The command \code{\textbackslash{}maketitle\{\}} tells \LaTeX{} to typeset the title of your document.

\item
  The command \code{\textbackslash{}section} tells \LaTeX{} to start a document section.
  The \emph{starred version} \code{\textbackslash{}section*}, which we use here, results in an unnumbered section.
  \inst{Go ahead and give this section a name by replacing \enquote{Section name}.}

\item
  The \code{pset} environment is defined by the \code{ccpset} package which we loaded in the preamble.
  It is a \enquote{list-like environment}; it consists of a sequence of \code{\textbackslash{}problem}s and \code{\textbackslash{}exercise}s, each of which creates another item in the list.
  \inst{Go ahead and write some text after the \code{\textbackslash{}problem}.}
\end{itemize}

That's it!
You now have a complete \LaTeX{} file, ready to go.

If you're using Write\LaTeX{}, your document should have been automatically recompiling as you went; the rendered version is conveniently displayed in a pane to the right of the editing window.
On the other hand, if you're using desktop software, you'll need to recompile manually.

Congratulations!
You \LaTeX{}ed!

\section{Your first real problem set}
\label{s:firstpset}
Now it's time to use \LaTeX{} to typeset one of your problem sets.
It's not much different than what we just did in \cref{s:firstdoc}!
We just need to talk about how to add more problems and sections.
\inst{Open a new copy of the problem set template} and we'll begin.

\subsection*{Sections}
If you want to add more sections, you can do so by typing \code{\textbackslash{}section*\{Name of Section\}} at the place where you want the section to break.
This must happen \emph{outside} the \code{pset} environment, so be sure to \code{\textbackslash{}end\{pset\}} before you \code{\textbackslash{}section} and to \code{\textbackslash{}begin\{pset\}} after.

You can also create nested sections.
\LaTeX{} has support for three layers of these (\code{section}, \code{subsection}, and \code{subsubsection}), but realistically you should probably stay at \code{subsection} and above for normal documents.

\subsection*{More problems}
To add more problems or exercises to your problem set, just call \code{\textbackslash{}problem} or \code{\textbackslash{}exercise} for each one.
Be sure to give each one a name as an argument; for example, you could indicate Exercise 4.3(b) by typing \code{\textbackslash{}exercise\{4.3(b)\}}.
Just be sure to put all your \code{\textbackslash{}problem} and \code{\textbackslash{}exercise} commands inside a \code{pset} environment!

\subsection*{Example file}
There's an example file with lots more details available as part of the \code{ccpset} package.
Just open \filename{ccpset-example.tex} in your editor to take a look.

\subsection*{Important information}
A few miscellany you should know before you dive in:
\begin{itemize}
\item
  \LaTeX{} will ignore single line breaks and treat double line breaks as paragraph breaks.
  It's common practice to put a line break at the end of each sentence to make your source code more readable.
  Thus, you might type something like this:
\begin{verbatim}
This is the first sentence of a paragraph.
This is the second sentence of the same paragraph, which we ought to make a little longer.

This sentence is part of another paragraph.
\end{verbatim}
  and obtain this:
  \begin{quote}
    This is the first sentence of a paragraph.
    This is the second sentence of the same paragraph, which we ought to make a little longer.

    This sentence is part of another paragraph.
  \end{quote}

\item
  Relatedly, \LaTeX{} will treat any number of spaces as the same thing as a single space.
  Thus, you might type something like this:
\begin{verbatim}
This is a sentence.
This    is     another     sentence.
\end{verbatim}
  and obtain this:
  \begin{quote}
    This is a sentence.
    This    is     another     sentence.
  \end{quote}

\item
  If you get errors from the compiler instead of your output, don't despair!
  There are a few easy-to-fix problems that are the source of the vast majority of these errors.
  There's a great discussion of these on the \LaTeX{} \href{https://en.wikibooks.org/wiki/LaTeX}{Wikibook}'s \href{https://en.wikibooks.org/wiki/LaTeX/Errors_and_Warnings}{Errors and Warnings} page.
\end{itemize}

\subsection*{Mathematical typesetting}
This is all great if you just need to typeset text, but you probably want to use \LaTeX{} for your math homework!
As illustrated in \cref{eq:statsthing,eq:greens}, \LaTeX{} is great at this, but it does require a bit of explanation.
Check out the Math Typesetting Guide for more information.
\end{document}