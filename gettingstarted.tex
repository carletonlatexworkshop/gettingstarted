\documentclass{article}
\usepackage{cctexexample}

\title{Getting started with \LaTeX{}}
\author{Carleton College \LaTeX{} workshop}
\date{}

\newcommand*{\code}[1]{\texttt{#1}}
\newcommand*{\filename}[1]{\texttt{#1}}
\newcommand*{\inst}[1]{\textbf{#1}}

\DeclareMathOperator{\dd}{d}

\usepackage{marvosym}

\begin{document}
\maketitle

Hello, and welcome to \LaTeX{}!
If you're just hearing about \LaTeX{} for the first time, you should read this document once through from beginning to end before you do anything with your computer.
Otherwise, you may want to skip into specific sections.
\begin{itemize}
\item
  \Cref{s:prelim} is a brief introduction to the philosophy and core concepts of \LaTeX{} as a system for producing documents.
\item
  \Cref{s:yourcomp} contains instructions for setting up a \LaTeX{} environment on your own computer.
\item
  \Cref{s:firstdoc} describes how to write a minimal compileable document and make a PDF out of it.
\end{itemize}


\section{Preliminaries}
\label{s:prelim}
You've probably used a word processor (like Microsoft's Word or OpenOffice Writer) to produce documents in the past.
\LaTeX{} is also a way to produce documents, but it's very different.

A typical \LaTeX{} document looks something like this:
\begin{lstlisting}[frame=single,caption=Basic \LaTeX{} example,label=lst:basicex, language=TeX]
  \documentclass{article}
  \title{Article title}
  \author{Author name}
  \begin{document}
  \maketitle
  This is an example of a document file.
  \end{document}
\end{lstlisting}

It's code!
\LaTeX{} is a \emph{structured markup language}, sort of like HTML; you build a document by writing some text which is \enquote{marked up} with information which describes that text's structure.
This code is the interpreted by a computer program which renders it into a nicely typeset PDF.

\LaTeX{} has a lot of great advantages:
\begin{itemize}
\item
  It's great at handling complex typesetting tasks like mathematical formulas.
  Check this out:
  \begin{equation}
    R = \frac{ \sum_{i=1}^{n} \pbrac{x_{i} - \bar{x}} \pbrac{y_{i} - \bar{y}} }{ \sqrt{\sum_{i = 1}^{n} \pbrac{x_{i} - \bar{x}}^{2} \sum_{i = 1}^{n} \pbrac{y_{i} - \bar{y}}^{2}} }
    \label{eq:statsthing}
  \end{equation}
  or this:
  \begin{equation}
    \label{eq:greens}
    \iint_{\Sigma} \cbrac*{ \pbrac*{ \frac{\partial R}{\partial y} - \frac{\partial Q}{\partial z}} \dd y \dd z +\pbrac*{\frac{\partial P}{\partial z} - \frac{\partial R}{\partial x}} \dd z \dd x  + \pbrac*{\frac{\partial Q}{\partial x} - \frac{\partial P}{\partial y}} \dd x \dd y}
    = \oint_{\partial \Sigma} \cbrac{P \dd x + Q\dd y + R \dd z}
  \end{equation}
  Pretty slick, huh?

\item
  It separates content from style.
  For example, to start a new section of a document, you just type \verb|\section{Name of section}|, and to produce text which is \emph{emphasized}, you just type \verb|\emph{emphasized}|.
  This means that you never have to worry about remembering how to style any particular element, and it makes writing documents \emph{much faster} once you get the hang of it.

\item
  It produces beautiful documents.
  The underlying \TeX{} typesetting engine is extremely powerful and professional-grade, producing output light-years ahead of what's typical from word processors.
  Your eyes will thank you!

\item
  It has powerful cross-referencing and other document tools.
  For example, I can refer to \cref{eq:greens} above simply by typing \verb|\label{eq:greens}| in the code for the equation and then \verb|\ref{eq:greens}| down here.
  The numbering is handled automatically, so I can move things around or even change the numbering scheme without breaking it.

\item
  It's highly portable.
  The source code you write is plain text, so you can email it, post it on a website, or even print it out and type it back in without messing anything up.
  The rendered documents are PDF, which will look the same on any computer in the world.
  No more worrying about what versions of Word everyone in your group is using!
\end{itemize}

Of course, it also has some disadvantages and costs.
\begin{itemize}
\item
  You'll need to install some software to \enquote{compile} \LaTeX{} source, and you'll probably need a better text editor than Notepad or TextEdit.
  We'll take care of that shortly.

\item
  It's hard to use \LaTeX{} for page layout.
  It will format your documents the way it thinks is best, and you're going to have a bad time if you try to tell it what to do.
  Go with the flow and focus on your content; your documents will look great!
  (If you have specific page-layout needs, like a newsletter or event poster, \LaTeX{} probably isn't the right software for that task.)

\item
  You don't know how to use it.
  \textbf{Yet.}
\end{itemize}

\section{Getting your computer set up for \LaTeX{}}
\label{s:yourcomp}
In order to use \LaTeX{} to write documents, you need two pieces of software: a \LaTeX{} environment, which \enquote{compiles} your source files into PDF output, and a suitable text editor.
Both of these are set up and ready to go on the lab computers in the CMC, so one option is just to fire one of those up and skip ahead to the next section.

Another option is to do all your \LaTeX{} work \enquote{in the cloud}.
There are several services which allow you to write your \LaTeX{} documents in an editor inside your browser, have them compiled by the server, and then rendered right there in real time.
This can be a great option, but as with any cloud service you should be careful about where your files are---you don't want your comps paper suddenly disappearing if a company goes out of business!
Some of the more popular options in this category are \href{http://www.sharelatex.com}{Share\LaTeX{}}, \href{http://www.writelatex.com}{Write\LaTeX{}}, and \href{http://www.latexlab.org}{\LaTeX{}Lab}.
A discussion of their features is beyond the scope of this guide, but you should definitely check them out!
If you're using one of these or otherwise already have a \LaTeX{} setup, feel free to skip ahead to the next section.

OK, you're still here, so you must want to get this going on your own computer.
Great!
The first step is to download and install \href{http://www.tug.org/texlive/acquire-netinstall.html}{\TeX{} Live}.
This is a \LaTeX{} \enquote{distribution}, which includes the compilers and all the support files you need to work with \LaTeX{} documents; it's available for Windows and OSX (along with most major Linux distributions, but you should install on those using their package managers).
You also need a suitable text editor, but \TeX{} Live comes with the very capable \TeX{}works, so you're all set there.
The full install is several gigabytes, so be sure to do this on a fast Internet connection.
Come on back when you're done.

Now that \TeX{} Live is installed, your computer is ready to edit and compile \LaTeX{} documents.
Go ahead and open up \TeX{}works from wherever new applications go in your operating system.
You should see a blank text-editing window.
You're ready to begin!

\section{Your first \LaTeX{} document}
\label{s:firstdoc}
\inst{Grab a copy of the Carleton problem set style and template.}
This should consist of three files:
\begin{itemize}
\item
  \filename{ccpset.sty}, a \enquote{style file} that tells \LaTeX{} how to style and structure your document;
\item
  \filename{ccpset.tex}, a stripped-down template ready for you to use for your problem sets; and
\item
  \filename{ccpset-example.tex}, an example document prepared using the style and template with lots of documentation and explainations.
\end{itemize}

\inst{Put the style file and template in the same directory, then open the template in your text editor.}
Here's what you should see:

\lstinputlisting[frame=single, caption=The Carleton problem set template, label=ccpsetempty, language=TeX]{homework/ccpset.tex}

Let's take a quick tour of the important features of this file and edit it as we go.
The first part is the \emph{preamble}, which configures \LaTeX{} and tells it about your document.

\begin{itemize}
\item
  Any line that starts with a \code{\%} is a \emph{comment}.
  The \LaTeX{} compiler will ignore these lines; they're only there to be read by humans.

\item
  Anything that starts with a \code{\textbackslash} is a \emph{command}.
  Each command can take some \emph{arguments}, which are contained in curly braces (\code{\{} and \code{\}}), and possibly an \emph{optional argument}, which is contained in square braces (\code{[} and \code{]}).

\item
  The command \code{\textbackslash{}documentclass} on the first line sets the document's \emph{class}; in this case, it is set to \code{article}.
  Document classes are used to set up things like page size, pagination, header formatting, and other large-scale things.
  The optional argument \code{twoside} tells \code{article} to set things up for double-sided printing, which changes how margins are set up.

\item
  The command \code{\textbackslash{}usepackage} tells \LaTeX{} to load a \emph{package}, which can execute lots of code on your behalf.
  The \code{ccpset} package configures things for a Carleton problem set, and the \code{lmodern} package fixes some problems with the default \LaTeX{} fonts.

\item
  The commands \code{\textbackslash{}title}, \code{\textbackslash{}author}, \code{\textbackslash{}date}, and \code{\textbackslash{}prof} set the information that will be used to make the document's title header.
  \inst{Go ahead and replace their arguments with some of your own; for example, replace \enquote{Author} with your own name.}
  Be sure to leave the opening (\code{\{}) and closing (\code{\}}) braces!
\end{itemize}

The second part of the file is the \emph{body}, where you write the actual text of your document.
\begin{itemize}
\item
  The entire body is wrapped in the \enquote{document} \emph{environment}, which starts with \code{\textbackslash{}begin\{document\}} and ends with \code{\textbackslash{}end\{document\}}.

\item
  The command \code{\textbackslash{}maketitle\{\}} tells \LaTeX{} to typeset the title of your document.

\item
  The command \code{\textbackslash{}section} tells \LaTeX{} to start a document section.
  The \emph{starred version} \code{\textbackslash{}section*}, which we use here, results in an unnumbered section.
  \inst{Go ahead and give this section a name by replacing \enquote{Section name}.}

\item
  The \code{pset} environment is defined by the \code{ccpset} package which we loaded in the preamble.
  It is a \enquote{list-like environment}; it consists of a sequence of \code{\textbackslash{}problem}s and \code{\textbackslash{}exercise}s, each of which creates another item in the list.
  \inst{Go ahead and write some text after the \code{\textbackslash{}problem}.}
\end{itemize}

That's it!
You now have a complete \LaTeX{} file, ready to go.

Of course, you wouldn't want to submit it like this---what you have right now is \emph{code}, not a beautiful document.
We still need to \emph{compile} the source code you've prepared into a typeset PDF.
How you do this will depend on your choice of text editor.
\begin{itemize}
\item
  If you're using the \TeX{}works editor, there should be a small green button with the standard \enquote{Play} symbol (\Forward) next to a drop-down box.
  Make sure the box says \enquote{pdf\LaTeX{}}, then \inst{hit the button!}

\item
  % TODO: Instructions for whatever they have in the labs.
\end{itemize}

If everything went according to plan, you should now be looking at a nicely-typeset PDF of your problem set.
Congratulations!
You \LaTeX{}ed!
\end{document}